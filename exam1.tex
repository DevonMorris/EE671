\documentclass{exam}
\usepackage{ dsfont }
\usepackage{ mathtools }
\usepackage{ commath }
\usepackage[final]{ pdfpages }

\name{Timothy Devon Morris}
\course{EC En 671}
\term{Fall 2018}
\examnum{1}

\begin{document}
  I certify that the solutions to this exam resprent my own owrk, and that I did not consult with any other individual about the exam.
  \vspace*{50px}
  \begin{problem}
    Let $\mathcal{M}$ be the set of $3 \times 3$ complex matrices, and let
    \[
      M_1 = 
      \begin{bmatrix}
        1 & j & 0 \\
        0 & 1 & j \\
        0 & j & 0 
      \end{bmatrix}
    \]
    \[
      M_2 = 
      \begin{bmatrix}
        1 + j & 1 + j & 1 + j \\
        2 & 2 & 2 \\
        0 & 0 & 0
      \end{bmatrix}
    \]
    \[
      M_3 = 
      \begin{bmatrix}
        1 - j & 4 & 5 \\
        1 - 2j & 6 & 7 \\
        1 - 3j & 8 & 9
      \end{bmatrix}
    \]
    \begin{parts}
      \part
      Show that $\langle A, B\rangle = tr(B^HA)$, where $tr(\cdot)$ is the trace of a matrix, is a valid inner product on $\mathcal{M}$.
      \part
      Show that $\{M_1, M_2, M_3\}$ are linearly independent.
      \part
      Find three orthonormal matrices $N_1, N_2, N_3$ such that $span\{N_1, N_2, N_3\} = span\{M_1, M_2, M_3\}$. If you use Matlab of Python, include your code.
    \end{parts}
  \end{problem}

  \begin{solution}
    \begin{parts}
      \part It suffices to show that the four axioms of an inner product space hold. Consider the matrices $X,Y,Z \in \mathcal{M}$ and $\alpha \in \mathds{C}$.
      \begin{parts}[R]
        \part 
        \[
          \langle X, Y \rangle = tr(Y^HX) = \overline{tr((Y^HX)^H)} = \overline{tr(X^HY)} = \overline{\langle Y, X \rangle}
        \]
        \part We can exploit the fact that the trace is linear
        \[
          \langle \alpha X, Y\rangle = tr(Y^H(\alpha X)) = tr(\alpha Y^HX) = \alpha tr(Y^HX) = \alpha \langle X, Y\rangle
        \]
        \part We can exploit the fact that the trace in linear
        \[
          \begin{aligned}
            \langle X + Y, Z\rangle &= tr(Z^H(X + Y)) = tr(Z^HX + Z^HY) \\
                                    &= tr(Z^HX) + tr(Z^HY) = \langle X, Z\rangle + \langle Y, Z\rangle
          \end{aligned}
        \]
        \part 
        \[
          \langle X, X\rangle = tr(X^HX) = \sum_{i=1}^n [X^HX]_{ii} =
          \sum_{i=1}^n\sum_{k=1}^n \overline{X_{ki}}X_{ki} = \sum_{i,k} \left| X_{ki} \right|^2 \geq0
        \]
        Note that in order for $\langle X, X\rangle = 0$ it must be that $X_{ki} = 0$ for all $i,k$. Therefore, $\langle X, X\rangle = 0$ if and only if $X = 0$. 
      \end{parts}
      \part Consider the weighted sum $\sum_{i=1}^3 \alpha_i M_i$ for $\alpha_i \in \mathds{C}$. We have that
      \[
        \sum_{i=1}^3 \alpha_iM_i = 
        \begin{bmatrix}
          \alpha_1 + \alpha_2(1 + j) + \alpha_3(1-j) & \alpha_1 j + \alpha_2(1 + j) + 4 \alpha_3 & \alpha_2(1 + j) + 5\alpha_3 \\
          2\alpha_2 + \alpha_3(1 - 2j) & \alpha_1 + 2\alpha_2 + 6\alpha_3 & \alpha_1 j + 2\alpha_2 + 7 \alpha_3 \\
          \alpha_3(1-3j) & \alpha_1j + 8\alpha_3 & 9 \alpha_3 
        \end{bmatrix}
      \]
      At this point, note that if we can find any three entries satisfying the the linear independence condition, we have that $M_1, M_2, M_3$ are linearly independent. Consider the following matrix which is the first column of $\sum_{i=1}^3 \alpha_iM_i$
      \[
        \begin{bmatrix}
          1 & 1+j & 1-j \\
          0 & 2 & 1 - 2j \\
          0 & 0 & 1 - 3j
        \end{bmatrix}
        \begin{bmatrix}
          \alpha_1 \\
          \alpha_2 \\
          \alpha_3
        \end{bmatrix}
      \]
      Since the determinant of this matrix is $3 - 9j \neq 0$ we have that that the matrices $M_1, M_2, M_3$ are linearly independent.
      \part
      I implemented this in a jupyter notebook and used the modified gram schmidt algorithm
      \includepdf[pages=-]{exam1prob1c.pdf}

    \end{parts}
  \end{solution}
\end{document}
