\documentclass{exam}
\usepackage{ dsfont }
\usepackage{ mathtools }
\usepackage{ commath }
\usepackage{ amsmath }
\usepackage[final]{ pdfpages }

\name{Timothy Devon Morris}
\course{EC En 671}
\term{Fall 2018}
\examnum{2}

\begin{document}
  I certify that the solutions to this exam repesent my own work, and that I did not consult with any other individual about the exam.
  \vspace*{50px}
  \begin{problem}
     Consider the discrete time state space equations
     \[
         \begin{aligned}
             x_{k+1} &= Ax_k \\
             y_k &= Cx_k,
         \end{aligned}
     \]
     with initial condition $x_0$, where $x_k \in \mathds{R}^n$ and $y_k$ is a scalar.
     \begin{parts}
         \part Show that $y_k = CA^kx_0$.
         \part Consider the linear operator $\mathcal{A}: \mathds{R}^n \rightarrow \ell_2$
         \[
             \mathcal{A}[x_0] = \left( Cx_0, CAx_0, CA^2x_0, CA^3x_0, \dots \right)
         \]
         Find the adjoint $\mathcal{A}^*: \ell_2 \rightarrow \mathds{R}^n$ of $\mathcal{A}$.
         \part Suppose that you collect an infinite sequence of data $e_k = y_k - Cx_k = y_k - CA^kx_0$. Find the formula for the least squares solution for the initial condition $x_0$ that minimizes 
         \[
             \sum_{k=0}^{\infty} \norm{y_k - CA^kx_0}^2.
         \]
     \end{parts}
  \end{problem}
  \begin{solution}
      \begin{parts}
        \part We will proceed by induction. For the base case $k = 0$ we have that         
        \[
            y_0 = Cx_0 = CIx_0 = CA^0x_0
        \]
        Furthermore, we have that
        \[
            x_1 = Ax_0
        \]
        Now we will assume that $x_{k} = A^{k}x_{0}$, now we have that
        \[
            x_{k+1} = Ax_{k} = AA^{k}x_{0} = A^{k+1}x_0
        \]
        and thus we have
        \[
            y_{k+1} = Cx_{k+1} = CA^{k+1}x_0
        \]
        Therefore, by induction we have that
        \[
            y_k = CA^kx_0
        \]
        for all $k \in \mathds{N}$
        \part
        let $x_0 \in \mathds{R}^n$ and $y \in \ell_2$. We want to find $\mathcal{A}^*$ such that
        \[
            \langle \mathcal{A}x_0, y \rangle_{\ell_2} = \langle x_0, \mathcal{A}^*y \rangle_{\mathds{R}^n}
        \]
        So expanding the the first expression expressions, we have 
        \[
            \langle \mathcal{A}x_0, y \rangle = \sum_{k=0}^{\infty} CA^kx_0y_k = \sum_{k=0}^{\infty} x_0^T(A^k)^TC^Ty_k = x_0^T \sum_{k=0}^{\infty} y_k(A^k)^TC^T
        \]
        Now let $\mathcal{A}^*: \ell_2 \rightarrow \mathds{R}^n$ be 
        \[
            \mathcal{A}^*[y] = \sum_{k=0}^{\infty}y_k(A^k)^TC^T
        \]
        And we have that
        \[
            \langle \mathcal{A}x_0, y\rangle = x_0^T\mathcal{A}^*[y] = \langle \mathcal{A}^*[y], x_0 \rangle = \langle x_0, \mathcal{A}^*[y]\rangle
        \]
        Since $x_0$ and $y$ are arbitrary, we have found the adjoint.
        Note: there is an implicit assumption in the definition of $\mathcal{A}$ that it actually maps into $\ell_2$. A sufficient condition for this would be stability of the linear system.
        \part
        Note, this optimization can be rewritten as
        \[
            \underset{x \in \mathds{R}^n}{\text{minimize}}: \norm{y - \mathcal{A}[x_0]}_{\ell_2}^2
        \]
        We now note that $\mathcal{A}^*\mathcal{A}: \mathds{R}^n \rightarrow \mathds{R}^n$ and is given by 
        \[
            \begin{aligned}
                \mathcal{A}^*\mathcal{A}[x_0] &= \mathcal{A}^* \left[ \left( Cx_0, CAx_0, CA^2x_0,CA^3x_0,\dots \right) \right] \\
                                              &= \sum_{k=0}^{\infty} (CA^kx_0)(A^k)^TC^T \\
                                              &= \sum_{k=0}^{\infty} (CA^k)^T(CA^k)x_0
            \end{aligned}
        \]
        Or we can write this as a matrix
        \[
            \mathcal{A}^*\mathcal{A} = \sum_{k=0}^{\infty} (CA^k)^T(CA^k)
        \]
        Under some assumptions (specifically that the observability matrix is full rank) we can define an inverse operation
        \[
            (\mathcal{A}^*\mathcal{A})^{-1} = \left( \sum_{k=0}^{\infty} (CA^k)^T(CA^k) \right)^{-1}
        \]
        Where the inverse is taken as a matrix inverse. By the fredholm alternative and the projection theorem, we have that
        \[
            x_0 = \left( \sum_{k=0}^{\infty} (CA^k)^T(CA^k) \right)^{-1} \sum_{k=0}^{\infty} (CA^k)^Ty_k
        \]
        is the minimizer.
      \end{parts}
  \end{solution}

  \begin{problem}
      Suppose that $A \in \mathds{C}^{m \times n}$ and that $\text{rank}(A) = r < \text{min}(m,n)$. Show that the QR factorization of $A$ can be written as
      \[
          A = 
          \begin{bmatrix}
              Q_1 & Q_2
          \end{bmatrix}
          \begin{bmatrix}
              R_1 & R_2 \\
              0_{(m-r) \times r} & 0_{(m-r) \times (n-r)}
          \end{bmatrix}
      \]
      Where $Q_1 \in \mathds{C}^{m \times r}$ and $Q_2 \in \mathds{C}^{m \times (m-r)}$, $R_1 \in \mathds{C}^{r \times r}$, and $R_2 \in \mathds{C}^{r \times (n-r)}$. Furthermore, show that $\mathcal{R}(A) = \text{span}(Q_1)$ and $\mathcal{N}(A^H) = \text{span}(Q_2)$.
  \end{problem}

  \begin{solution}

  Now that we have found our decomposition, we wish to show that $\mathcal{R}(A) = \mathcal{R}(Q_1)$. We will first show that $\mathcal{R}(A) \subset \mathcal{R}(Q_1)$. Let $y \in \mathcal{R}(A)$, thus we have an $x$ such that $y = Ax$. So we have that
      \[
         y = 
         \begin{bmatrix}
             Q_1R_1 & Q_1R_2 \\
         \end{bmatrix}
         \begin{bmatrix}
             x_1 \\
             x_2
         \end{bmatrix}
         = 
         Q_1R_1x_1 + Q_1R_2x_2 = Q_1(R_1x_1 + R_2x_2)
      \]
      Thus $y \in \mathcal{R}(Q_1)$ and $\mathcal{R}(A) \subset \mathcal{R}(Q_1)$. Now we wish to show that $\mathcal{R}(Q_1) \subset \mathcal{R}(A)$. Let $y \in \mathcal{R}(Q_1)$, thus we have an $x$ such that $y = Q_1x$.
  \end{solution}

\end{document}
