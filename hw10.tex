\documentclass{homework}
\usepackage{ dsfont }
\usepackage{ mathtools }
\usepackage{ commath }
\usepackage[final]{ pdfpages }

\name{Timothy Devon Morris}
\course{EC En 671}
\term{Fall 2018}
\hwnum{10}

\begin{document}

\begin{problem}[14-15]
  A white noise signal $f[t]$ is passed through a system with impulse response $\{.5, -1, -2, 1, .5\}$. Write a program to identify the system using the LMS algorithm. Try your experiment where the variance of $f[t]$ is $\sigma_f^2 = 0.1$ and the following variations
  \begin{parts}
    \part The LMS filter has five coefficients
    \part The LMS filter has ten coefficients
    \part The LMS filter has three coefficients.
  \end{parts}
  Determine experimentally the range of $\mu$ for which the adaptive filter converges.
\end{problem}

\begin{solution}
  \includepdf[pages=-]{hw10prob1.pdf}   
\end{solution}

\begin{problem}[18-2a]
  Let $h(x) = x_1^2 - x_2$ 
  \begin{parts}
    \part
    Make a plot of the "surface" $h(x) = 0$
    \part
    Determine a parameterization $x(\xi)$ of points on the surface.
    \part
    Plot the vector $dx(\xi)/d\xi$ and the vector $\nabla h(x)$ at the point $x = (2,4)$, Show that these vectors are orthogonal.
  \end{parts}
\end{problem}

\begin{solution}
  \includepdf[pages=-]{hw10prob2.pdf}
\end{solution}

\begin{problem}[18-7]
 Consider the problem
 \[
   \begin{aligned}
     &\text{maximize: } 2x_1x_2 + x_2x_3 + x_1x_3 \\
     &\text{subject to: } x_1 + x_2 + x_3 = 3
   \end{aligned}
 \]
 Determine the solution $(x_1, x_2, x_3, \lambda)$. Compute the matrix $\mathbf{L}$, and determine if your solution is a minumum, maximum, or neither.
\end{problem}

\begin{solution}
 We begin by constructing the lagrangian
 \[ L(x_1, x_2, x_3, \lambda) = 2x_1x_2 + x_2x_3 + x_1x_3 + \lambda(x_1 + x_2 + x_3 - 3)\]
 By the first order necessary conditions, we have that
 \[
   \begin{bmatrix}
     2x_2 + x_3 + \lambda \\
     2x_1 + x_3 + \lambda \\
     x_2 + x_1 + \lambda \\
     x_1 + x_2 + x_3
   \end{bmatrix}
   = 
   \begin{bmatrix}
     0 \\
     0 \\
     0 \\
     3
   \end{bmatrix}
 \]
 As a linear system, we have
 \[
   \begin{bmatrix}
     0 & 2 & 1 & 1 \\
     2 & 0 & 3 & 1 \\
     1 & 1 & 0 & 1 \\
     1 & 1 & 1 & 0
   \end{bmatrix}
   \begin{bmatrix}
     x_1 \\
     x_2 \\
     x_3 \\
     \lambda
   \end{bmatrix}
   =
   \begin{bmatrix}
     0 \\
     0 \\
     0 \\
     3
   \end{bmatrix}
 \]
 Solving this linear system gives us
 \[
   \begin{aligned}
     x_1 &= 1.5 \\
     x_2 &=  1.5 \\
     x_3 &= 0 \\
     \lambda &= -3
   \end{aligned}
 \]
 So now we have that 
 \[
   \mathbf{L}
   =
   \begin{bmatrix}
     0 & 2 & 1\\
     2 & 0 & 1\\
     1 & 1 & 0
   \end{bmatrix}
 \]
 Which is indefinite, therefore the this point is neither a maximum or minimum.
\end{solution}

\end{document}
